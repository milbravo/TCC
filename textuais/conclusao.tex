% ------------------------------------------------------------------------------
% Conclusão
% ------------------------------------------------------------------------------

\chapter{Conclusão}\label{chap:conclusao}

Os conceitos e boas práticas da Engenharia de Sistemas (ES) são versáteis e aplicáveis em diferentes tipos de ambientes e organizações. O trabalho realizado demonstra que,
mesmo em cenários bastante específicos e fora dos padrões tradicionais, há ganhos significativos ao analisar o ciclo de vida sob a ótica da ES. Essa análise possibilita uma
tomada de decisão mais embasada e assertiva nas mudanças sugeridas, que, por sua vez, mostraram-se eficazes na redução de problemas previamente identificados — especialmente na
definição e validação de conceito e na rastreabilidade dos componentes do sistema.

A abordagem de revisar e buscar melhorias no ciclo de vida de um serviço de software por meio das práticas da ES reforça a diversidade de aplicações possíveis, extrapolando os
limites da indústria tradicional e dos projetos de engenharia clássicos. O próprio \cite{incoseHandbook} e a obra de \cite{kossiakoff2020systems} apresentam exemplos de aplicação
da ES em domínios como segurança cibernética e inteligência artificial.

A autonomia concedida para a implementação das modificações no ciclo de vida foi crucial para o êxito do trabalho. Sem isso, não teria sido possível a análise de dados reais da equipe.
Ser o membro com maior senioridade no time técnico, aliado ao número reduzido de pessoas na equipe, facilitou a implantação das mudanças e a aceitação por parte dos demais
membros — algo que poderia representar um desafio maior em equipes mais numerosas ou com profissionais mais experientes já acostumados com o processo anterior, mesmo que este
apresentasse limitações. Vale destacar que o novo ciclo de vida foi aplicado em apenas dois sistemas reais voltados para a área de clientes, além de seu uso no próprio aplicativo
desenvolvido neste trabalho. No entanto, nenhum dos projetos havia sido concluído até o momento, o que limita a maturidade dos dados disponíveis para uma avaliação mais ampla e
representativa do serviço de criação de sistemas.

Os resultados obtidos reforçam a importância de um ciclo de vida bem estruturado e, especialmente, das fases de Definição de Conceito e Desenvolvimento Avançado — cruciais em projetos
com escopo fechado. No caso de sistemas com prazos de desenvolvimento atipicamente curtos, como os analisados, essas fases tornam-se ainda mais críticas, já que qualquer erro na
estimativa de esforço representa um impacto proporcionalmente elevado no prazo final de entrega. A melhoria na distribuição do esforço por tarefa, observada neste trabalho, contribui
para a redução desses erros ao aumentar a granularidade das ações a serem executadas. Isso também gera maior visibilidade para o gerente de projetos, permitindo um gerenciamento de
riscos mais preciso. Além disso, a nova estrutura operacional possibilitou a introdução de ferramentas de inteligência artificial na rotina da equipe, alinhando-se às tendências de
mercado voltadas à automação de tarefas repetitivas.

A ferramenta desenvolvida para rastreabilidade do sistema demonstrou ser um recurso eficaz para melhorar a acurácia na estimativa de esforço. O aumento dessa estimativa após sua adoção
indica a redução de percepções excessivamente otimistas quanto ao esforço requerido por tarefas consideradas simples. Isso evita a alocação indevida de múltiplas tarefas simultâneas
para um único recurso, contribuindo para uma gestão de carga de trabalho mais realista. Ademais, a rastreabilidade melhora o planejamento técnico para manutenções e inclusão de
melhorias em sistemas existentes, apoiando diretamente o gerenciamento de riscos técnicos.

Como sugestões para trabalhos futuros, destaca-se a necessidade de analisar o impacto das mudanças no prazo efetivo de conclusão dos projetos. Outra oportunidade de evolução está na
ampliação da hierarquia de elementos dentro do aplicativo de rastreabilidade, incluindo funcionalidades e processos de mais alto nível, oferecendo uma visão ainda mais integrada do
sistema. A coleta de uma base de dados maior e mais variada também seria valiosa, especialmente para aprofundar a análise do esforço relacionado à rastreabilidade em múltiplos cenários.

Por fim, concluímos que estudos sobre aplicações não convencionais da Engenharia de Sistemas, como o realizado neste trabalho, são fundamentais para a identificação de novos ganhos e
oportunidades de adaptação metodológica. Esta pesquisa contribui para o corpo de conhecimento da área ao demonstrar, de forma concreta, como os métodos e processos da ES podem ser
aplicados com precisão na identificação de falhas no ciclo de vida e nas fases em que ocorrem, servindo de base para modificações eficazes de processos. Trata-se de uma demonstração
prática do valor da Engenharia de Sistemas como ferramenta de transformação organizacional, inclusive em domínios que não são tradicionalmente associados à engenharia clássica.

