% ------------------------------------------------------------------------------
% Conclusão
% ------------------------------------------------------------------------------

\chapter{Conclusão}\label{chap:conclusao}
	
	Os conceitos e boas práticas da Engenharia de Sistemas são versáteis e podem ser aplicados em diversos tipos de ambientes e organizações. O trabalho realizado mostra
	que mesmo em cenários bem específicos e fora dos padrões tradicionais, há ganhos na análise do ciclo de vida sob a ótica da ES. Essa análise permite uma tomada de decisao mais embasada e acertiva
	nas mudanças sugeridas, que por sua vez, mostram-se efetivas em reduzir os problemas iniciais identificados, na definicão e validação do conceito e na rastreabilidade dos componentes do sistema.

	{\color{red}Contextualização}
	
	A autonomia concedida para a aplicação das modificações do ciclo de vida foram de extrema importância para o trabalho, sem isso não seria possível a análise de dados reais da equipe. Ser o membro da equipe
	técnica com mais senioridade do time e o número reduzido de pessoas facilitou a implantação das mudanças e a aceitação do restante da equipe, que poderia ser um desafio considerável
	em times maiores ou com pessoas mais experientes já ambientadas com o processo antigo, mesmo este sendo problemático. Um ponto a se destacar é que o novo ciclo de vida foi aplicado em apenas dois sistemas reais
	para área clientes e ao próprio aplicativo desenvolvido nesse trabalho, e não chegou a ser concluído ainda para nenhum dos sistemas. Logo a representatividade dos dados ainda não tem um nível grande de maturidade
	em uma visão macro do serviço de criação de sistemas.

	Os resultados encontrados reforçam a importância de um ciclo de vida bem estruturado e ademais, a importância das fases de Definicão do Conceito e Desenvolvimento Avançado, principalmente em projetos de escopo fechado.
	No caso dos sistemas desenvolvidos que possuem um tempo de projeto atipicamente curto, essas duas fases tem um impacto maior ainda, pois qualquer erro na estimativa de esforço significa um alto percentual de atraso na entrega final.
	Uma melhor distribuição de esforço por tarefa como foi encontrado, ajuda na redução desse erro pois aumenta a granularidade das ações que devem ser executadas. Isso contribui ainda com uma maior visibilidade ao gerente de projetos, para realizar um trabalho de gerenciamento de riscos de projeto mais acertivo.
	Um outro ganho operacional foi a possibilidades da aplicação de IA na rotina de trabalho, que vai de encontro com os movimentos do mercado de buscar a redução do tempo empenhado em tarefas que podem ser feitas por esse tipo de ferramenta.

	Uma ferramenta para manutenção da rastreabilidade do sistema, como o aplicativo desenvolvido, juntamente com o resultado do aumento da estimativa de esforço após a utilização do mesmo, implica em uma redução de uma percepção de esforço muito otimista.
	Isso minimiza uma carga excessiva de trabalho ao alocar essas tarefas, inicialmente simples, para serem executas em conjunto com outras, e depois ela se tornar mais trabalhosa que o esperado.
	Essa visão da rastreabilidade melhora também o gerenciamento de riscos técnicos ao se planejar uma manuentenção ou na inclusão de uma melhoria em um sistema já existente.

	Para trabalhos futuros podem ser observados o efeito no prazo efetivo para a conclusão dos projetos de sistemas com essas mudanças no ciclo de vida. É um ponto de avanço incluir além dos elementos do sistema no aplicativo de rastreabilidade, as funcionalidades relacionadas acima deles, para tornar mais completa a hierarquia.
	Uma análise com mais dados seria de bom proveito também, principalmente na comparação de esforço na parte da rastreabilidade, explorando mais cenários ou dados reais que possam ser coletados futuramente.

	{\color{red}Conclusão final}







	Escrever um bom capítulo de conclusão em trabalhos acadêmicos envolve sintetizar os principais achados da pesquisa, refletir sobre o significado desses resultados, e sugerir direções futuras. Aqui estão algumas diretrizes para estruturar este capítulo:

	\begin{itemize}
		\item Resumo dos Principais Achados: Comece recapitulando os principais resultados da pesquisa. Destaque como esses resultados atendem aos objetivos do estudo ou respondem às perguntas de pesquisa.
		
		\item Contextualização: Discuta a importância dos resultados no contexto do campo de estudo. Isso inclui como seus achados se alinham ou divergem de estudos anteriores.
		
		\item Reflexão Crítica: Inclua uma autoavaliação da pesquisa, abordando limitações e como elas podem ter afetado os resultados. Isso demonstra integridade acadêmica e compreensão das nuances da pesquisa.
		
		\item Implicações Práticas e Teóricas: Explique as implicações dos seus resultados para a prática, teoria ou política. Isso mostra a relevância e o valor do seu trabalho.
		
		\item Sugestões para Pesquisas Futuras: Baseando-se nas limitações e nos achados da sua pesquisa, sugira áreas para futuras investigações. Isso ajuda a avançar o campo de estudo.
		
		\item Conclusão Final: Termine com uma conclusão forte que reafirme a contribuição do seu trabalho para o campo de estudo. Isso pode incluir uma declaração poderosa sobre o significado dos seus achados ou uma visão para o futuro da área de pesquisa.
		
		\item Produção Bibliográfica: Se aplicável, liste as publicações geradas a partir da sua pesquisa. Isso pode incluir artigos, apresentações em conferências, ou outros materiais acadêmicos.
	\end{itemize}
