% ------------------------------------------------------------------------------
% Introdução
% ------------------------------------------------------------------------------

\chapter{Introdução}\label{chap:introducao} % Inglês: Introduction

	Ao longo dos anos vivenciando a experiência no mercado de trabalho em diferentes 
	empresas, é possível notar a dificuldade em mapear, seguir e otimizar processos 
	nas organizações. Um fator agravante é quando o processo não é simplesmente 
	para definir operações administrativas, mas também para definir um modelo de 
	serviço prestado,{\color{blue} como consultorias, treinamentos e criação de aplicativos. Além 
	dos processos para a concepção do produto os materiais ofertados, a prestação do 
	serviço possui processos próprios que afetam no relacionamento com as partes interessadas
	bem como no resultados final a ser entregue.} O serviço em questão pode ser resumido no desenvolvimento de 
	ferramentas ou soluções para estruturação, automação e/ou digitalização de processos dentro 
	de organizações. Outro ponto que vale ser ressaltado, é a necessidade de um vasto conhecimento 
	em arquitetura de soluções, de ferramentas e técnicas de desenvolvimento para 
	viabilização do serviço e sua manutenção.

	{\color{green} Seria legal começar mencionar outros exemplos e depois mencionar o serviço que tem a ver com o seu trabalho. E não precisa falar que vai ser detalhado mais adiante.}
	{\color{green} Esse parágrafo pode vir no final do anterior.}

	Vale notar que o mercado cria a necessidade do serviço, e este 
	surge de forma orgânica e na maioria dos casos sem uma definição adequada. Um ciclo 
	de vida do sistema de interesse, seja ele um produto ou um serviço, é essencial para 
	padronizar as etapas. A documentação dessas etapas pode ser estabelicida de acordo com 
	os padrões da Engenharia de Sistemas (ES). 
	{\color{green} Acho que ainda pode ser mantido 
	um tom mais geral, como por exemplo: ``Um ciclo de vida do sistema de interesse, seja ele um produto ou um serviço, é essencial para padronizar as etapas. A documentação dessas etapas pode ser estabelicida de acordo com os padrões da Engenharia de Sistemas (ES).''}

	Quando um ciclo de vida não é bem definido, deficiências operacionais nas integrações 
	e relacionamentos entre as partes interessadas são comuns de serem observados. 
	Onde mais pode se destacar essa deficiência é na área de gerenciamento de requisitos e 
	rastreabilidade do sistema. Quando um ciclo de vida não possui nenhuma documentação de 
	requisitos ou rastreabilidade, isso pode afetar tanto as estimativas de esforço e tempo 
	para os desenvolvimentos e possíveis mudanças, quanto deixa uma grande incerteza de impacto 
	em futuras manutenções. Ao resolver um problema ou instabilidade, podem ser gerados outros 
	que só serão notados pelos usuários finais do sistema.
	
	{\color{green} No jeito mais generalista, poderia ser: ``Quando um ciclo de vida não é bem definido, deficiências operacionais nas integrações e relacionamentos entre as partes interessadas são comuns de serem observados.''}
	{\color{green} De uma forma mais generalista: ``Onde mais pode se destacar essa deficiência é na área de gerenciamento de requisitos e rastreabilidade do sistema. Quando um ciclo de vida não possui nenhuma documentação de requisitos ou rastreabilidade, isso pode afetar tanto as estimativas de esforço e tempo para os desenvolvimentos e possíveis mudanças, quanto deixa uma grande incerteza de impacto em futuras manutenções. Ao resolver um problema ou instabilidade, podem ser gerados outros que só serão notados pelos usuários finais do sistema.''}

	Alterações e inclusão de requisitos nas fases finais de desenvolvimento ou depois 
	do sistema desenvolvido são muito mais custosas que se estes tivessem sido 
	refinados e definidos no início. E sem a capacidade de analisar a rastreabilidade do 
	sistema, como dito anteriormente, as estimativas têm pouco fundamento e 
	refletem pouco a realidade. Logo, prazos são mal calculados, métricas são extraídas 
	de maneira errônea e o gerenciamento da equipe fica prejudicado. 

	Manter o ciclo de vida do sistema de interesse atualizado e alinhado à realidade 
	ajuda a reduzir custos tanto na concepção quanto na operação do sistema. Isso ocorre 
	de forma direta, otimizando os recursos utilizados por meio de processos e etapas bem 
	definidas, e de forma indireta, diminuindo o retrabalho e o tempo gasto em atividades desnecessárias.

	Tem sido uma tendência em diversas empresas a criação de times, setores ou áreas focadas na digitalização 
	e automação de processos, rotinas ou atividades repetitivas na empresa.
	As áreas de negócios {\color{blue}que são colocadas como clientes para esse serviço} abrangem equipes 
	como marketing, finanças, pagamentos, segurança da informação, tesouraria, infraestrutura de tecnologia, 
	qualidade, recursos humanos e diversos outros times que desempenham atividades de escritório. Essas equipes 
	podem ser alvo da iniciativa de digitalização e são consideradas os principais interessados, atuando como 
	clientes do serviço prestado.

	Este trabalho se propõe a estudar e analisar um sistema dentro de uma empresa multinacional, com 
	diversos ramos de atuação e ativos geradores de receita. A partir do mapeamento do ciclo de vida 
	atual, propostas de revisão e melhoria são investigadas e elaboradas para abordar as deficiências 
	operacionais nas integrações e relacionamentos com as partes interessadas. Em seguida será proposta 
	uma ferramenta para garantir a rastreabilidade do sistema e o gerenciamento dos requisitos de forma 
	mais integrada e dinâmica.
	
	{\color{green} Aqui então você pode começar a falar de uma maneira mais direta o que você se propõe a estudar. Então, você pode começar da seguinte forma: ``Este trabalho se propõe a estudar e analisar um sistema dentro de uma empresa multinacional, com diversos ramos de atuação e ativos geradores de receita. A partir do mapeamento do ciclo de vida atual, propostas de revisão e melhoria são investigadas e elaboradas para abordar as deficiências operacionais nas integrações e relacionamentos com as partes interessadas. Em seguida será proposta uma ferramenta para garantir a rastreabilidade do sistema e o gerenciamento dos requisitos de forma mais integrada e dinâmica.''}

	{\color{green} Essa próxima frase pode vir antes de você começar a falar diretamente do seu trabalho. ``Tem sido uma tendência em diversas empresas a criação de times, setores ou áreas focadas na digitalização e automação de processos, rotinas ou 
	atividades repetitivas na empresa.''} 
	
	{\color{green} Essa parte aqui eu não sei muito bem aonde encaixar. ``O surgimento do time de atuação veio dessa tendência, dentro da grande área de ``Digitalização Global'' da organização foi montada uma equipe com o objetivo de atender todos esses focos de trabalho, com atendimento disponível para todas as áreas de negócio da América do Sul.''} 

	{\color{green} Esse parágrafo pode vir depois de ``Tem sido uma tendência em diversas empresas a criação...'' e antes do começo da fala direta do seu trabalho.}

	O ciclo de vida a ser analisado descreve um serviço interno desta empresa a clientes de digitalização de processos. Esse serviço é o desenvolvimento de uma solução 
	tecnológica que resolva algum problema ou automatize algum processo trazido pelo 
	cliente. É feita uma análise do contexto e realizada uma proposta de resolução 
	para o problema, e em seguida se inicia o desenvolvimento caso isso seja decidido. 
	Em alguns casos, é necessário ainda o suporte à solução desenvolvida, dependendo 
	da complexidade e volume de utilização. 

	O time de desenvolvimento trabalha numa estrutura de fábrica de aplicativos ou fábrica de software, 
	com poucos desenvolvedores experientes e generalistas, para serem bem 
	engajados com todas as possibilidades a serem exploradas. Esse serviço é repetido 
	a cada nova solução desenvolvida para os times clientes, novos ou não. 
	Diversas tecnologias são utilizadas durante o desenvolvimento, sendo definidas de 
	acordo com a necessidade de cada situação. Entretanto, as duas principais 
	tecnologias são as ferramentas \textit{low code}\footnote{Ferramentas low code são plataformas de desenvolvimento que permitem a criação de aplicativos e sistemas com pouca ou nenhuma necessidade de programação manual. Elas utilizam interfaces visuais, como arrastar e soltar componentes, e configurações pré-definidas para simplificar o processo de desenvolvimento.} da empresa Microsoft, conhecidas como 
	ferramentas da \textit{Power Platform}, e códigos em Python para execução de RPAs (do 
	inglês ``Robot Process Automation''). 

	\section{Objetivos Geral e Específicos}\label{sec:introducao:objetivos}

		Podemos destacar como objetivos gerais do trabalho realizado a implementação de 
		técnicas e processos de Engenharia de Sistemas para otimizar a prestação de 
		serviços de software de curta duração. {\color{red} Acho que poderia ser escrito da seguinte forma: ``O objetivo geral deste trabalho é melhorar a prestação de serviços de software de curta duração por meio da reforma de processos dentro do seu ciclo de vida.''}
		Já como objetivos específicos podem ser destacados: 
		\begin{itemize}
			\item O mapeamento da situação atual do serviço prestado e do produto entregue.
			\item A identificação de oportunidades de melhoria no presente cenário. 
			\item Melhora da definição e levantamento de requisitos no processo existente.
			\item Elaboração de uma ferramenta para o gerenciamento de rastreabilidade e requisitos.
		\end{itemize}
 		
	\section{Contribuições e Originalidade}\label{sec:introducao:contribuicoes}

		Tanto os aspectos de digitalização e otimização de processos quanto o uso de tecnologias \textit{low code} são relativamente novos no que diz respeito à padrões definidos
		de desenvolvimento, arquitetura ou boas práticas.
	
		O uso de técnicas de ES é interessante pois ao contrário do movimento normal no desenvolvimento de soluções de software que seguem métodos ágeis com muitas iterações e entregas
		de valor, o processo a ser estudado segue um modelo mais próximo do tradicional modelo de cascata. Entretanto, são conduzidos modelos cascata de curtíssima duração onde os prazos
		máximos de conclusão variam de 3 a 4 meses, com 3 ou 4 projetos sendo desenvolvidos simultâneamente.

		Dessa forma, ao longo do trabalho é desenvolvido um "tailoring" dos conhecimentos da ES para esse contexto e situação de desenvolvimento.
		{\color{red} Acho que ficaria melhor da seguinte forma: ``Dessa forma, os processos técnicos de ES são adaptados para esse contexto e situação de desenvolvimento.''}
	
	\section{Organização do Trabalho}\label{sec:introducao:organizacao}

		O restante deste trabalho está organizado da seguinte forma: no Capítulo \ref{chap:revisao} é apresentada a revisão bibliográfica necessária para o entendimento do trabalho; no Capítulo \ref{chap:humanidades} são apresentados os aspectos socioeconômicos; no Capítulo \ref{chap:metodologia} é apresentada a metodologia utilizada para a realização do trabalho. {\color{red} Mais pra frente aqui você vai mencionar os capítulos de resultado e conclusão.}
		
