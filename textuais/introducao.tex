% ------------------------------------------------------------------------------
% Introdução
% ------------------------------------------------------------------------------

\chapter{Introdução}\label{chap:introducao} % Inglês: Introduction

	Ao longo dos anos vivenciando a experiência no mercado de trabalho em diferentes 
	empresas, é possível notar a dificuldade em mapear, seguir e otimizar processos 
	nas organizações. Um fator agravante é quando o processo não é simplesmente 
	para definir operações administrativas, mas também para definir um modelo de 
	serviço prestado. O serviço em questão pode ser resumido no desenvolvimento de 
	ferramentas para estruturação, automação e/ou digitalização de processos dentro 
	de organizações e será detalhado mais adiante no texto.

	Outro ponto que vale ser ressaltado, é a necessidade de um vasto conhecimento 
	em arquitetura de soluções e de ferramentas e técnicas de desenvolvimento para 
	viabilização do serviço e sua manutenção. 

	Primeiramente, vale notar que o mercado cria a necessidade do serviço, e este 
	surge de forma orgânica e na maioria dos casos sem uma definição adequada. No 
	caso, será definido o ciclo de vida do sistema de interesse que constitui esse 
	serviço, a fim de padronizar as etapas e documentar de acordo com os padrões da 
	engenharia de sistemas. 

	Mesmo sem esse ciclo de vida definido já são percebidas deficiências operacionais 
	nas integrações e relacionamentos entre as partes interessadas do sistema de 
	interesse. Onde mais se destaca essa deficiência é na parte de 
	gerenciamento de requisitos e rastreabilidade do sistema. No ciclo de vida atual 
	não há nenhuma documentação de requisitos ou rastreabilidade, e isso afeta tanto 
	as estimativas de esforço e tempo para os desenvolvimentos e possíveis mudanças, 
	quanto deixa uma grande incerteza de impacto nas manutenções que possam ser 
	necessárias. Ao resolver um problema ou instabilidade, podem ser gerados outros 
	que só serão notados pelos usuários finais do sistema. 

	Alterações e inclusão de requisitos nas fases finais de desenvolvimento ou depois 
	do sistema desenvolvido são muito mais custosas que se estes tivessem sido 
	refinados e definidos no início. E sem a capacidade de analisar a rastreabilidade do 
	sistema, como dito anteriormente, as estimativas têm pouco fundamento e 
	refletem pouco a realidade. Logo, prazos são mal calculados, métricas são extraídas 
	de maneira errônea e o gerenciamento da equipe fica prejudicado. 

	Manter o ciclo de vida do sistema de interesse atualizado e fiel à realidade faz com 
	que custos sejam reduzidos durante a concepção e operação do sistema. De forma 
	direta na otimização dos recursos empregados que só é possível com processos
	etapas bem definidas, quando indireta na redução de retrabalho e tempo gasto em 
	atividades desnecessárias.  

	Após o mapeamento do ciclo de vida atual, propostas de revisão e melhoria serão 
	feitas para melhor tratamento da deficiência operacional citada anteriormente. Em 
	seguida será proposta uma ferramenta para garantir a rastreabilidade do sistema 
	e o gerenciamento dos requisitos de forma mais integrada e dinâmica. 

	O estudo de caso e análise do sistema de interesse acontece em uma empresa 
	multinacional, com diversos ramos de atuação e ativos geradores de receita. No 
	entanto, a respeito do escopo de trabalho do time em questão, esses detalhes de 
	produtos e mercado não são diretamente relevantes para o sistema de interesse. 

	Como dito anteriormente, o mercado gerou o serviço e o sistema de interesse a 
	ser trabalhado. Tem sido uma tendência em diversas empresas a criação de times, 
	setores ou áreas focadas na digitalização e automação de processos, rotinas ou 
	atividades repetitivas na empresa. O surgimento do time de atuação veio dessa 
	tendência, dentro da grande área de “Digitalização Global” da organização foi 
	montada uma equipe com o objetivo de atender todos esses focos de trabalho, com 
	atendimento disponível para todas as áreas de negócio da américa do sul. 

	Com áreas de negócios, são englobados times de marketing, finanças, 
	pagamentos, segurança da informação, tesouraria, infraestrutura de tecnologia, 
	qualidade, recursos humanos e vários outros times que trabalham em atividades 
	de escritório que podem ser alvo da equipe de digitalização. Esses são os principais 
	interessados, e considerados os clientes do serviço prestado. 

	O sistema de interesse a ser estudado é o serviço interno prestado a esses clientes 
	citados anteriormente. Esse serviço é o desenvolvimento de uma solução 
	tecnológica que resolva algum problema ou automatize algum processo trazido pelo 
	time cliente. É feita uma análise do contexto e realizada uma proposta de resolução 
	para o problema, e em seguida se inicia o desenvolvimento caso isso seja decidido. 
	Em alguns casos, é necessário ainda o suporte à solução desenvolvida, dependendo 
	da complexidade e volume de utilização. 

	O time trabalha numa estrutura de fábrica de aplicativos ou fábrica de software, 
	com poucos desenvolvedores experientes e generalistas, para serem bem 
	engajados com todas as possibilidades a serem exploradas. Esse serviço é repetido 
	a cada nova solução desenvolvida para os times clientes, novos ou não. 
	Diversas tecnologias são utilizadas durante o desenvolvimento, sendo definidas de 
	acordo com a necessidade de cada situação. Entretanto, as duas principais 
	tecnologias são as ferramentas “low code” da empresa Microsoft, conhecidas como 
	ferramentas da “Power Platform”, e códigos em Python para execução de RPAs (do 
	inglês “robot process automation”). 

	\section{Objetivos Geral e Específicos}\label{sec:introducao:objetivos}

		Podemos destacar como objetivos gerais do trabalho realizado a implementação de 
		técnicas e processos de engenharia de sistemas para otimizar a prestação de um 
		serviço de software. 
		Já como objetivos específicos podem ser destacados: 
		\begin{itemize}
			\item O mapeamento da situação atual do serviço prestado e do produto entregue.
			\item A identificação de oportunidades de melhoria no presente cenário. 
			\item Melhora da definição e levantamento de requisitos no processo existente. 
		\end{itemize}
 		
 		

	\section{Contribuições e Originalidade}\label{sec:introducao:contribuicoes}

		Descrever as contribuições do trabalho, indicando o que o trabalho
		propõe de novo ou diferente em relação ao estado da arte. As contribuições
		devem ser claras e objetivas, indicando o que o trabalho agrega ao conhecimento
		existente.
	
	\section{Organização do Trabalho}\label{sec:introducao:organizacao}

		Descrever a organização do trabalho, indicando o conteúdo de cada capítulo
		e a relação entre eles. A organização do trabalho deve ser clara e
		coerente, de forma a facilitar a compreensão do leitor.
