% ------------------------------------------------------------------------------
% Introdução
% ------------------------------------------------------------------------------

\chapter{Introdução}\label{chap:introducao} % Inglês: Introduction

	Ao longo dos anos, vivenciando a experiência no mercado de trabalho em diferentes 
	empresas, é possível notar a dificuldade em mapear, seguir e otimizar processos 
	nas organizações. Um fator agravante é quando o processo se propõe não somente a definir operações administrativas, mas também a estabelecer um modelo de 
	serviço prestado, como consultorias, treinamentos e criação de aplicativos. Além 
	dos processos para a concepção do produto, a prestação do 
	serviço possui processos próprios que afetam o relacionamento com as partes interessadas,
	bem como o resultado final a ser entregue. O serviço em questão pode ser resumido no desenvolvimento de 
	ferramentas ou soluções para estruturação, automação e/ou digitalização de processos dentro 
	de organizações. Outro ponto que vale ser ressaltado é a necessidade de um vasto conhecimento 
	em arquitetura de soluções, de ferramentas e técnicas de desenvolvimento para 
	viabilização do serviço e sua manutenção.

	Nota-se que o mercado cria a necessidade do serviço, que 
	surge de forma orgânica e, na maioria dos casos, sem uma definição adequada. Um ciclo 
	de vida do sistema de interesse, seja ele um produto ou um serviço, é essencial para 
	padronizar as etapas. A documentação dessas etapas pode ser estabelecida de acordo com 
	os padrões da Engenharia de Sistemas (ES). 

	Quando um ciclo de vida não é bem definido, deficiências operacionais nas integrações 
	e nos relacionamentos entre as partes interessadas são comuns de serem observadas. 
	Onde mais se destaca essa deficiência é na área de gerenciamento de requisitos e 
	rastreabilidade do sistema. Quando um ciclo de vida não possui nenhuma documentação de 
	requisitos ou rastreabilidade, isso pode afetar tanto as estimativas de esforço e tempo 
	para os desenvolvimentos e possíveis mudanças quanto deixar uma grande incerteza de impacto 
	em futuras manutenções. Ao resolver um problema ou instabilidade, podem ser gerados outros 
	que só serão notados pelos usuários finais do sistema.

	Alterações e inclusão de requisitos nas fases finais de desenvolvimento ou depois 
	do sistema desenvolvido são muito mais custosas do que se estes tivessem sido 
	refinados e definidos no início. E, sem a capacidade de analisar a rastreabilidade do 
	sistema, como dito anteriormente, as estimativas têm pouco fundamento e 
	refletem pouco a realidade. Logo, prazos são mal calculados, métricas são extraídas 
	de maneira errônea e o gerenciamento da equipe fica prejudicado. 

	Manter o ciclo de vida do sistema de interesse atualizado e alinhado à realidade 
	ajuda a reduzir custos tanto na sua concepção quanto na sua operação. Isso ocorre 
	de forma direta, otimizando os recursos utilizados por meio de processos e etapas bem 
	definidas, e de forma indireta, diminuindo o retrabalho e o tempo gasto em atividades desnecessárias.

	Tem sido uma tendência, em diversas empresas, a criação de times, setores ou áreas focadas na digitalização 
	e automação de processos, rotinas ou atividades repetitivas na empresa.
	As áreas de negócios que são colocadas como clientes para esse serviço abrangem equipes 
	como marketing, finanças, pagamentos, segurança da informação, tesouraria, infraestrutura de tecnologia, 
	qualidade, recursos humanos e diversos outros times que desempenham atividades de escritório. Essas equipes 
	podem ser alvo da iniciativa de digitalização e são consideradas os principais interessados, atuando como 
	clientes do serviço prestado.

	Este trabalho se propõe a estudar e analisar um sistema dentro de uma empresa multinacional, com 
	diversos ramos de atuação e ativos geradores de receita. A partir do mapeamento do ciclo de vida 
	atual, propostas de revisão e melhoria são investigadas e elaboradas para abordar as deficiências 
	operacionais nas integrações e nos relacionamentos com as partes interessadas. Em seguida, será proposta 
	uma ferramenta para garantir a rastreabilidade do sistema e o gerenciamento dos requisitos de forma 
	mais integrada e dinâmica.

	O ciclo de vida a ser analisado descreve um serviço interno dessa empresa, prestado pelo setor de digitalização.
	Esse serviço consiste no desenvolvimento de uma solução tecnológica que resolva algum problema ou automatize algum processo trazido pelo 
	cliente. É feita uma análise do contexto e realizada uma proposta de resolução 
	para o problema, e em seguida se inicia o desenvolvimento, caso isso seja decidido. 
	Em alguns casos, é necessário ainda o suporte à solução desenvolvida, dependendo 
	da complexidade e do volume de utilização. 

	O time de desenvolvimento trabalha numa estrutura de fábrica de aplicativos ou fábrica de software, 
	com poucos desenvolvedores experientes e generalistas, para que estejam bem 
	engajados com todas as possibilidades a serem exploradas. Esse serviço é repetido 
	a cada nova solução desenvolvida para os times clientes, novos ou não. 
	Diversas tecnologias são utilizadas durante o desenvolvimento, sendo definidas de 
	acordo com a necessidade de cada situação. Entretanto, as duas principais 
	tecnologias são as ferramentas \textit{low code}\footnote{Ferramentas low code são plataformas de desenvolvimento
	que permitem a criação de aplicativos e sistemas com pouca ou nenhuma necessidade de programação manual e infraestrutura dedicada. Elas utilizam
	interfaces visuais, como arrastar e soltar componentes, e configurações pré-definidas para simplificar o processo de desenvolvimento.}
	da empresa Microsoft, conhecidas como ferramentas da \textit{Power Platform}, e códigos em Python para execução de RPAs (do 
	inglês \textit{Robot Process Automation}).


	\section{Objetivos Geral e Específicos}\label{sec:introducao:objetivos}

	O objetivo geral deste trabalho é melhorar a prestação de serviços de software de curta duração por meio da reforma de processos dentro do seu ciclo de vida.

	De forma específica, pretende-se, primeiramente, realizar o mapeamento da situação atual do serviço prestado e do produto entregue.
	Em seguida, busca-se identificar oportunidades de melhoria no cenário existente, com foco na eficiência e qualidade das entregas.
	Também é objetivo aprimorar a definição e validação do conceito aplicado ao processo em uso, tornando-o mais claro e confiável.
	Por fim, este trabalho visa ainda a elaboração de uma ferramenta que auxilie no gerenciamento de rastreabilidade e dependências ao longo do desenvolvimento dos serviços.

 		
	\section{Contribuições e Originalidade}\label{sec:introducao:contribuicoes}

	As contribuições deste trabalho situam-se na interseção entre a digitalização e otimização de processos e o uso de tecnologias \textit{low code},
	áreas que ainda carecem de padrões consolidados em termos de desenvolvimento, arquitetura e boas práticas. A proposta se destaca por
	aplicar técnicas de Engenharia de Sistemas (ES) em um cenário atípico para a disciplina, caracterizado por projetos de curta duração e execução simultânea.

	Ao contrário do fluxo comum no desenvolvimento de software, que adota metodologias ágeis com múltiplas iterações e entregas incrementais,
	o processo analisado neste estudo aproxima-se de um modelo sequencial, similar ao tradicional modelo em cascata. Contudo, trata-se de uma versão extremamente
	condensada, com prazos máximos entre três e quatro meses e múltiplos projetos sendo conduzidos em paralelo, o que impõe desafios adicionais de rastreabilidade
	e validação de requisitos.

	Nesse contexto, os processos clássicos da Engenharia de Sistemas, como definição e validação de conceito, análise de requisitos, gerenciamento de interfaces
	e integração, são adaptados para atender às restrições de tempo e recursos, utilizando ferramentas de apoio como as da \textit{Power Platform}.

	A adaptação dos métodos de ES para ciclos curtos de desenvolvimento e sua aplicação prática com suporte de plataformas \textit{low code} representa uma contribuição
	original deste trabalho, tanto do ponto de vista metodológico quanto tecnológico. Além disso, este estudo demonstra a viabilidade da Engenharia de Sistemas em
	ambientes com forte demanda por entregas rápidas, oferecendo uma abordagem estruturada que mantém a rastreabilidade, consistência e qualidade dos produtos entregues.

		
	\section{Organização do Trabalho}\label{sec:introducao:organizacao}

		O restante deste trabalho está organizado da seguinte forma:

		No Capítulo~\ref{chap:revisao}, é apresentada a revisão bibliográfica necessária para o entendimento do tema, abordando os principais conceitos e
		definições relacionados à Engenharia de Sistemas, com ênfase no ciclo de vida de um sistema e nas diferentes arquiteturas que podem ser utilizadas para documentá-lo.

		No Capítulo~\ref{chap:humanidades}, são discutidos os aspectos e impactos socioeconômicos relacionados ao trabalho desenvolvido, considerando como esses fatores
		influenciam o ambiente organizacional, os colaboradores e a sociedade ao redor.

		O Capítulo~\ref{chap:metodologia} apresenta a metodologia empregada na realização do estudo, incluindo o levantamento do processo atual de prestação do serviço,
		a adequação às boas práticas da Engenharia de Sistemas, a identificação de problemas ao longo do ciclo de vida, as propostas de melhoria e a concepção da aplicação
		desenvolvida para gerenciar a rastreabilidade.

		
		{\color{red} Mais pra frente aqui você vai mencionar os capítulos de resultado e conclusão.}
		
