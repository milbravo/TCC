% ------------------------------------------------------------------------------
% Introdução
% ------------------------------------------------------------------------------

\chapter{Aspectos Socioeconômicos e Humanidades}\label{chap:humanidades} % Inglês: humanities

A digitalização de processos empresariais, aliada à automação de tarefas e ao uso de ferramentas \textit{low-code}, é uma estratégia indispensável para aumentar a competitividade, eficiência e segurança nas operações, conforme enfatizado por \cite{sebraeDigitalizacaoProcessos}. Estudos apontam que sua adoção pode:

\begin{itemize}
	\item \textbf{Elevar a produtividade}: Com ganhos de até 30\%, ao reduzir em até 90\% o tempo para tarefas repetitivas, permitindo que colaboradores se concentrem em atividades estratégicas.
	\item \textbf{Aumentar a eficiência}: A automação otimiza fluxos de trabalho, com melhorias de até 80\% na performance operacional, reduzindo prazos de entrega e aumentando a satisfação dos clientes.
	\item \textbf{Reduzir custos}: Economias podem chegar a 90\% no processamento de dados, 30\% na manutenção de equipamentos e 40\% na gestão documental.
	\item \textbf{Melhorar a experiência do cliente}: Processos digitais permitem respostas mais rápidas e personalizadas, podendo aumentar a receita em até 10\% e impulsionar o crescimento das empresas em 2,2 vezes.
	\item \textbf{Aumentar a segurança da informação}: Com controles mais rigorosos, a proteção de dados pode ser ampliada em até 50\%, reduzindo riscos e fortalecendo a confiabilidade.
\end{itemize}

A Engenharia de Sistemas fornece a base conceitual e metodológica para gerenciar a possível complexidade envolvida na transformação digital. Segundo \cite{incoseHandbook}, essa disciplina busca desenvolver soluções integradas e coerentes que levem em conta todos os domínios de um sistema — técnico, humano, organizacional e ambiental.

Nesse contexto, o uso de ferramentas \textit{low-code} e automação deve ser compreendido como parte de um ciclo de vida sistêmico, que abrange:

\begin{itemize}
	\item \textbf{Requisitos multidisciplinares}: Envolvendo áreas de negócio, tecnologia e stakeholders sociais.
	\item \textbf{Modelagem de processos}: Para representar fluxos e identificar pontos críticos de automação.
	\item \textbf{Validação e verificação}: Garantindo que as soluções automatizadas entreguem valor de forma confiável.
	\item \textbf{Integração e interoperabilidade}: Facilitada por abordagens modulares e por plataformas \textit{low-code} com suporte a APIs.
\end{itemize}

Com base na Engenharia de Sistemas, é possível aplicar princípios como o pensamento sistêmico, a modelagem orientada a funções e a arquitetura empresarial para guiar decisões tecnológicas alinhadas com os objetivos organizacionais e sociais.

A adoção dessas tecnologias impacta diretamente as dinâmicas de trabalho:

\begin{itemize}
	\item \textbf{Reconfiguração de papéis}: Profissionais de negócio assumem papéis mais ativos no ciclo de vida de sistemas, promovendo maior alinhamento entre necessidades e soluções.
	\item \textbf{Empoderamento dos colaboradores}: Usuários finais podem prototipar, testar e implantar soluções de forma autônoma, reduzindo burocracias.
	\item \textbf{Requalificação}: Com foco em habilidades analíticas, resolução de problemas e pensamento sistêmico.
\end{itemize}

Segundo \cite{mcKinseyAutomation}, até 45\% das atividades humanas podem ser automatizadas com tecnologias atuais. Isso não implica apenas substituição de postos de trabalho, mas principalmente transformação e valorização de funções com maior envolvimento cognitivo e decisório.

A Engenharia de Sistemas enfatiza a visão de ciclo de vida completo, o que inclui o descarte responsável, a adaptabilidade dos sistemas ao longo do tempo e o uso sustentável de tecnologias.

O uso conjunto de automação, plataformas \textit{low-code} e princípios da Engenharia de Sistemas representa uma convergência poderosa para transformar organizações de forma eficaz, inclusiva e sustentável. Essa abordagem interdisciplinar oferece não apenas ganhos operacionais, mas também impactos positivos sobre a estrutura organizacional, o desenvolvimento humano e a sustentabilidade ambiental.

