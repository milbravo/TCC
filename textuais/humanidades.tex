% ------------------------------------------------------------------------------
% Introdução
% ------------------------------------------------------------------------------

\chapter{Aspectos Socioeconômicos e Humanidades}\label{chap:humanidades} % Inglês: humanity
A digitalização de processos empresariais é uma estratégia indispensável para aumentar a competitividade, eficiência e segurança nas operações, conforme enfatizado por
\cite{sebraeDigitalizacaoProcessos}. Estudos apontam que sua adoção pode:
\begin{itemize}
	\item Elevar a produtividade: Com ganhos de até 30\%, ao reduzir em até 90\% o tempo para tarefas repetitivas, permitindo que colaboradores se concentrem em atividades estratégicas.
	\item Aumentar a eficiência: A automação otimiza fluxos de trabalho, com melhorias de até 80\% na performance operacional, reduzindo prazos de entrega e aumentando a satisfação dos
	clientes.
	\item Reduzir custos: Economias podem chegar a 90\% no processamento de dados, 30\% na manutenção de equipamentos e 40\% na gestão documental.
	\item Melhorar a experiência do cliente: Processos digitais permitem respostas mais rápidas e personalizadas, podendo aumentar a receita em até 10\% e impulsionar o crescimento
	das empresas em 2,2 vezes.
	\item Aumentar a segurança da informação: Com controles mais rigorosos, a proteção de dados pode ser ampliada em até 50\%, reduzindo riscos e fortalecendo a confiabilidade.
	\item A digitalização, portanto, transforma não apenas a operação interna das empresas, mas também sua interação com clientes e o mercado, assegurando competitividade e 
	sustentabilidade a longo prazo.
\end{itemize}
	