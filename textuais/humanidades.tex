% ------------------------------------------------------------------------------
% Introdução
% ------------------------------------------------------------------------------

\chapter{Aspectos Socioeconômicos e Humanidades}\label{chap:humanidades} % Inglês: humanities

A digitalização de processos empresariais, aliada à automação de tarefas e ao uso de ferramentas \textit{low-code}, é uma 
estratégia indispensável para aumentar a competitividade, eficiência e segurança nas operações, conforme enfatizado por 
\cite{sebraeDigitalizacaoProcessos}. Este estudo aponta que sua adoção pode elevar a produtividade, com ganhos de até 30\%, 
ao reduzir em até 90\% o tempo para tarefas repetitivas, permitindo que colaboradores se concentrem em atividades estratégicas. 
Além disso, contribui para aumentar a eficiência, uma vez que a automação otimiza fluxos de trabalho, com melhorias de até 80\% 
na performance operacional, reduzindo prazos de entrega e aumentando a satisfação dos clientes. Outro benefício significativo 
é a redução de custos, com economias que podem chegar a 90\% no processamento de dados, 30\% na manutenção de equipamentos e 
40\% na gestão documental. A digitalização também melhora a experiência do cliente, pois processos digitais permitem respostas 
mais rápidas e personalizadas, podendo aumentar a receita em até 10\% e impulsionar o crescimento das empresas em 2,2 vezes. 
Por fim, aumenta a segurança da informação, com controles mais rigorosos que podem ampliar a proteção de dados em até 50\%, 
reduzindo riscos e fortalecendo a confiabilidade.

A Engenharia de Sistemas fornece a base conceitual e metodológica para gerenciar a possível complexidade envolvida na 
transformação digital. Segundo \cite{incoseHandbook}, essa disciplina busca desenvolver soluções integradas e coerentes 
que levem em conta todos os domínios de um sistema — técnico, humano, organizacional e ambiental.

Nesse contexto, o uso de ferramentas \textit{low-code} e automação deve ser compreendido como parte de um ciclo de vida 
sistêmico, que abrange requisitos multidisciplinares, envolvendo áreas de negócio, tecnologia e stakeholders sociais; 
modelagem de processos, para representar fluxos e identificar pontos críticos de automação; validação e verificação, 
garantindo que as soluções automatizadas entreguem valor de forma confiável; e integração e interoperabilidade, 
facilitada por abordagens modulares e por plataformas \textit{low-code} com suporte a APIs.

Com base na Engenharia de Sistemas, é possível aplicar princípios como o pensamento sistêmico, a modelagem 
orientada a funções e a arquitetura empresarial para guiar decisões tecnológicas alinhadas com os objetivos 
organizacionais e sociais.

A adoção dessas tecnologias impacta diretamente as dinâmicas de trabalho: promove a reconfiguração de papéis, 
com profissionais de negócio assumindo funções mais ativas no ciclo de vida de sistemas, o que fortalece o 
alinhamento entre necessidades e soluções; facilita o empoderamento dos colaboradores, permitindo que usuários 
finais prototipem, testem e implantem soluções de forma autônoma, reduzindo burocracias; e exige requalificação, 
com foco no desenvolvimento de habilidades analíticas, resolução de problemas e pensamento sistêmico.

Segundo \cite{mcKinseyAutomation}, até 45\% das atividades humanas podem ser automatizadas com tecnologias atuais. Isso não implica apenas substituição de postos de trabalho, mas principalmente transformação e valorização de funções com maior envolvimento cognitivo e decisório.

A Engenharia de Sistemas enfatiza a visão de ciclo de vida completo, o que inclui o descarte responsável, a adaptabilidade dos sistemas ao longo do tempo e o uso sustentável de tecnologias.

O uso conjunto de automação, plataformas \textit{low-code} e princípios da Engenharia de Sistemas representa uma convergência poderosa para transformar organizações de forma eficaz, inclusiva e sustentável. Essa abordagem interdisciplinar oferece não apenas ganhos operacionais, mas também impactos positivos sobre a estrutura organizacional, o desenvolvimento humano e a sustentabilidade ambiental.

