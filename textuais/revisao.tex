% ------------------------------------------------------------------------------
% Revisão Bibliográfica
% ------------------------------------------------------------------------------

\chapter{Revisão Bibliográfica}\label{chap:revisao}

\section{Ciclo de Vida}

	A definição e criação de um ciclo de vida é uma das formas da Engenharia de Sistemas (ES) atuar no seu propósito de viabilizar o sucesso de um sistema ao
	mesmo tempo que otimiza a competição existente entre os objetivos das partes interessadas. Ao desmembrar o esforço total e definir os estágios, seus papéis novas
	caracteristíscas do sistema, seus critérios de completude, seus riscos existentes e ao fim tomar uma decisão, está sendo feito a criação do ciclo de vida.

	Entre cada estágio defido, há o que é chamado em inglês de ``decision gates'' onde é feita a análise do progresso, e como o nome sugere é tomada uma decisão
	quanto ao desenvolvimento do sistema.

	O ciclo de vida de um sistema é definido a partir de suas características e particularidades, de modo que seus estágios sejam inseridos para atender todas as suas
	necessidades. Os estágios podem aparecer mais de uma vez, serem executados sequencialmente ou paralelamente e serem inseridos a qualquer momento do ciclo de vida.

	Há casos em que o Sistema de Interesse(SoI do inglês ``System os Interests'') é parte de um Sistema de Sistemas(SoS, do inglês ``System os Systems''). Nesse caso,
	cada um tem seu próprio ciclo de vida, no geral em um SoS cada elemento do sistema terá seu próprio ciclo de vida, e o do SoS influencia no do SoI, de forma
	que sua evolução deve ser considerada quando olhado para o ciclo de vida do SoI.

	O ciclo de vida genérico trazido no \cite{incoseHandbook} nos mostra os seis estágios básicos existentes numa estruturação em "V" que busca mostrar de forma visual a aparição desses estágios
	ao longo do tempo, realçando também o possível paralelismo entre eles. Na figura \ref{} podemos ver uma representação do que foi mostrado no livro.

	O estágio de conceito: reprenta a parte exploratória de pesquisas e origens do reconheimento de uma necessidade, ou uma nova missão, ou uma nova capacidade de negócio, ou
	ainda a alteração de algum desses itens. Nesse estágio são explorados todos os fatores do sistema, desde mercado, ambientais, econômicos, recursos disponíveis e escopo de
	atuação, de forma que sejam definidos os limites do prolema a ser resolvido, as missões do sistema, onde ele será utilizado e seja feita uma análise do negócio, da missão e
	dos valores entregues. Para que o problema seja bem definido, são realizados os levantamentos dos requisito do sistema, das partes interessadas envolvidas e suas necessidades,
	e do espaço de solução, e assim pode ser derivado um custo inicial do esforço a ser empenhado e uma agenda prévia, que servem de base para o ciclo de vida. Algumas saídas
	ou resultados típicos desse estágio são documentos preliminares da arquitetura sistema, da viabilidade, dos requisitos, do design e novamente da agenda e esforço. Esse
	estágio é de extrema importância pois aqui o sistema é definido, mudanças podem surgir depois, mas representarão uma maior dificuldade de implementação devido a diversos
	fatores, como tempo ou custo. 

	O estágio de desenvolvimento: nesse estágio é definido um SoI que atende e vai de encontro com as necessidades e requisitos das partes interessadas, e que pode ser
	produzido, utilizado, suportado e descontinuado caso necessário. O objetivo principal dessa fase é definir um projeto base de engenharia que pode ser executado, sem buscar a 
	perfeição, mas atendendo às partes interessadas e respeitando os possíveis ``trade-offs'' previamente definidos nesse mesmo estágio. Nesse projeto base devem estar os
	requisitos, arquitetura, modelagens, documentação e planejamento para próximas fases que também podem ser vistos como saídas dessa fazer.

	O estágio de produção: nesse estágio o projeto base definido no estágio anterior sai do papel e dá lugar ao sistema de fato, que será testado e qualificado para ser
	colocado para utilização.

	O estágio de utilização: o início desse estágio se dá com a liberação do sistema ou parte dele para uso, incluindo os sistemas de apoio que são necessários para certas
	funcionalidades. Esse estágio comumente é o mais longo do ciclo de vida e é comum que mudanças e melhorias no SoI ocorram ao longo da utilização, lembrando sempre de fazer
	o gerenciamento dos riscos e documentação para garantir a integridade e manutenção do SoI.

	O estágio de suporte: segue paralelo ao estágio de utilização assim que alguma funcionalidade se torna disponível, no entanto o preparo e planejamento desse estágio pode 
	ser iniciado antes como a aquisição de sobresalentes. Nesse estágio que são percebidas as melhorias e mudanças que podem vir a ser implementadas durante a utilização.

	O estágio de descontinuação: acontece quando o sistema sai de operação e normalmente seu início dá fim aos estágios de utilização e produção, ou no máximo existe uma pequena
	sobreposição entre estes. Além de definir como será feito o descarte físico ou virtual das partes é nessa etapa que é feita um possivel análise de extensão de vida útil de parte do Sistema
	e o arquivamento de documentos importantes sobre o mesmo.

	Ainda no \cite{incoseHandbook}, são trazidos conceitos importante sobre os ``decision gates'' que coexistem entre os estágio dos ciclo de vida, tanto no início quanto no fim de cada estágio.
	Dentre os objetivos dos ``decision gates'' estão o acompanhamento da evolução da maturidade do sistema, a conferência dos critérios de saída ou entrada de um estágio,
	a análise de risco mediante à situação atual do sistema, e por fim uma tomada de decisão sobre o que será feito. Podendo haver um regresso no ciclo de vida, um avanço, uma pausa 
	ou até mesmo o cancelamento do projeto.

	É importante saber equilibrar a formalidade e frequência desses eventos, visto que eles envolvem diferentes partes interessadas, gestores e especialistas, e além disso as
	decisões devem ser guiadas por dados tomados nos estágios do ciclo de vida e nos artefatos que são gerados para esse momento. Isso evita considerações desnecessárias e 
	inadequadas que podem prejudicar futuramente.

	As três abordagens principais trazidas pelo livro para os ciclos de vida são, sequêncial, incremental e evolucionário. As principais características dessas três abordagens
	podem ser resumidas na tabela \ref{tab:revisao:ciclodevida:abordagens} apresentada.

	\begin{table}[!h]
		\centering
		\caption{Características das abordagens de um ciclo de vida}
		\begin{tabular}{cccc}
			\hline
			Abordadem & Requisitos definidos no ínicio & Iterações planejadas & Múltiplas instalações \\
			\hline
			Sequencial & Todos os requisitos & Apenas uma & Não\\
			Incremental & Todos os requisitos & Múltiplas & Potencialemnte\\
			Evolucionário & Parte dos requisitos & Múltiplas & Tipicamente\\
			\hline
		\end{tabular}
		\label{tab:revisao:ciclodevida:abordagens}
	\end{table}

	Durante a execução dos estágios do ciclo de vida várias tarefas são executadas, e para isso alguns processos precisam ser realizados para garantir a consistência das atividades. Um conjunto
	de processos é definido no livro e a execução de cada um deles varia de acordo com os estágios existentes no ciclo de vida do sistema.
	
\section{Arquitetura do Sistema}\label{sec:revisao:arqSistema}

	Como mencionado na seção \ref{sec:revisao:arqSistema} existem diferentes processos durante o ciclo de vida de um sistema ou projeto. Um deles é o 
	\textit{Processo de Definição da Arquitetura do Sistema}. A existência de um  \textit{Estilos de Arquitetura} é de extrema importância para que esse processo seja executado com êxito. Ele
	atua como um modelo, ou guia, para se construir a arquitetura do sistema. Os \textit{Estilos de Arquitetura} podem ser definidos com base nos ponto de vista da arquitetura, no elementos do
	sistema e seus relacionamentos, nas conexões, interfaces, mecanismos de interação e possíveis restrições.
	
	Além dos \textit{Estilos de Arquitetura}, outro conceito importante é o de \textit{Padrões de Arquitetura}. Eles são modelos simplificados mas completos no que diz respeito aos elementos do
	sistema e são reutilizáveis para diferentes tipos cenários. O uso de \textit{Padrões de Arquitetura} agiliza  a documentação, facilita a comunicação, promove o reúso, melhora a produtividade
	e eficiência e serve como um ponto de início para o desenvolvimento de novos sistemas.

	Como o conceito de arquitetura pode ser muito abrangente, o \cite{sebok2024} mostra três segmentações de arquitetura, a arquitetura funcional,	lógica e a física.

	A arquitetura funcional compreende as funcionalidades do sistema, ou seja, quais funções ou comportamentos aquele sistema executa ou possui em diferentes contextos para atingir os objetivos
	esperados.