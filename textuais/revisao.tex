% ------------------------------------------------------------------------------
% Revisão Bibliográfica
% ------------------------------------------------------------------------------

\chapter{Revisão Bibliográfica}\label{chap:revisao}

	Ao redigir uma revisão bibliográfica em trabalhos acadêmicos, é crucial adotar uma abordagem sistemática e crítica. Inicie identificando e selecionando fontes 
	relevantes que abordem diretamente o tema de pesquisa, priorizando publicações acadêmicas revisadas por pares, como artigos de periódicos, livros e conferências. 
	Uma boa prática é organizar a literatura em temas ou escolas de pensamento, facilitando a compreensão do leitor sobre o estado da arte e as lacunas existentes. 
	É importante também avaliar criticamente cada obra, discutindo sua contribuição para o campo, metodologias, resultados e limitações. A revisão deve ser escrita 
	de forma coesa, com transições suaves entre os trabalhos discutidos, e deve terminar destacando como a pesquisa atual se insere e contribui para o conhecimento 
	existente. Citando adequadamente todas as fontes, evita-se o plágio e reconhece-se o trabalho dos pesquisadores originais, além de fornecer ao leitor caminhos 
	para aprofundamento.

\section{Ciclo de Vida}

	A definição e criação de um ciclo de vida é uma das formas da Engenharia de Sistemas (ES) atuar no seu propósito de viabilizar o sucesso de um sistema ao
	mesmo tempo que otimiza a competição existente entre os objetivos das partes interessadas. Ao desmembrar o esforço total e definir os estágios, seus papéis novas
	caracteristíscas do sistema, seus critérios de completude, seus riscos existentes e ao fim tomar uma decisão, está sendo feito a criação do ciclo de vida.

	Entre cada estágio defido, há o que é chamado em inglês de \textit{decision gates} onde é feita a análise do progresso, e como o nome sugere é tomada uma decisão
	quanto ao desenvolvimento do sistema.

	O ciclo de vida de um sistema é definido a partir de suas características e particularidades, de modo que seus estágios sejam inseridos para atender todas as suas
	necessidades. Os estágios podem aparecer mais de uma vez, serem executados sequencialmente ou paralelamente e serem inseridos a qualquer momento do ciclo de vida.

	Há casos em que o Sistema de Interesse(SoI do inglês \textit{System os Interests}) é parte de um Sistema de Sistemas(SoS, do inglês \textit{System os Systems}). Nesse caso,
	cada um tem seu próprio ciclo de vida, no geral em um SoS cada elemento do sistema terá seu próprio ciclo de vida, e o do SoS influencia no do SoI, de forma
	que sua evolução deve ser considerada quando olhado para o ciclo de vida do SoI.

	O ciclo de vida genérico trazido no INCOSE nos mostra os seis estágios básicos existentes numa estruturação em "V" que busca mostrar de forma visual a aparição desses estágios
	ao longo do tempo, realçando também o possível paralelismo entre eles. Na figura \ref{} podemos ver uma representação do que foi mostrado no livro.

	O estágio de conceito: reprenta a parte exploratória de pesquisas e origens do reconheimento de uma necessidade, ou uma nova missão, ou uma nova capacidade de negócio, ou
	ainda a alteração de algum desses itens. Nesse estágio são explorados todos os fatores do sistema, desde mercado, ambientais, econômicos, recursos disponíveis e escopo de
	atuação, de forma que sejam definidos os limites do prolema a ser resolvido, as missões do sistema, onde ele será utilizado e seja feita uma análise do negócio, da missão e
	dos valores entregues. Para que o problema seja bem definido, são realizados os levantamentos dos requisito do sistema, das partes interessadas envolvidas e suas necessidades,
	e do espaço de solução, e assim pode ser derivado um custo inicial do esforço a ser empenhado e uma agenda prévia, que servem de base para o ciclo de vida. Algumas saídas
	ou resultados típicos desse estágio são documentos preliminares da arquitetura sistema, da viabilidade, dos requisitos, do design e novamente da agenda e esforço. Esse
	estágio é de extrema importância pois aqui o sistema é definido, mudanças podem surgir depois, mas representarão uma maior dificuldade de implementação devido a diversos
	fatores, como tempo ou custo. 

	O estágio de desenvolvimento: nesse estágio é definido um SoI que atende e vai de encontro com as necessidades e requisitos das partes interessadas, e que pode ser
	produzido, utilizado, suportado e descontinuado caso necessário. O objetivo principal dessa fase é definir um projeto base de engenharia que pode ser executado, sem buscar a 
	perfeição, mas atendendo às partes interessadas e respeitando os possíveis \textit{trade-offs} previamente definidos nesse mesmo estágio. Nesse projeto base devem estar os
	requisitos, arquitetura, modelagens, documentação e planejamento para próximas fases que também podem ser vistos como saídas dessa fazer.

	O estágio de produção: nesse estágio o projeto base definido no estágio anterior sai do papel e dá lugar ao sistema de fato, que será testado e qualificado para ser
	colocado para utilização.

	O estágio de utilização: aqui as partes do SoI "deixam" de existir e dão lugar a um sistema aprovado para ser utilizado.

	O estágio de suporte:

	O estágio de descontinuação:


