\usepackage[utf8]{inputenc} % Suporte a caracteres UTF-8
\usepackage{mathptmx} % Fonte Times New Roman (aproximada)
\usepackage{setspace} % Controle de espaçamento entre linhas
\usepackage{parskip} % Configuração de parágrafos (sem indentação e com espaço entre parágrafos)
\usepackage{indentfirst} % Indenta o primeiro parágrafo de cada seção
\usepackage{geometry} % Configuração das margens
\usepackage{titling} % Configurações de título e autor
\usepackage{emptypage} % Remove números e cabeçalhos de páginas vazias
\usepackage{ragged2e} % Alinhamento justificado à esquerda e à direita
\usepackage{hyphenat} % Suporte a hifenização avançada
\usepackage[table,xcdraw]{xcolor} % Suporte a cores em tabelas e desenhos
\usepackage{graphicx} % Inclusão de gráficos e imagens
\usepackage[algoruled, portuguese, linesnumbered]{algorithm2e} % Algoritmos com numeração de linhas e estilo específico
\usepackage[hidelinks]{hyperref} % Hiperlinks sem molduras coloridas
\usepackage[round,colon,sort]{natbib} % Citações bibliográficas com estilo específico
\usepackage[nottoc,notlot,notlof]{tocbibind} % Inclui "Referências" no índice
\usepackage{amsmath, amssymb, amsfonts, amsthm, mathtools} % Pacotes AMS para matemática avançada
\usepackage{enumitem} % Personalização de listas enumeradas
\usepackage{bm} % Fonte negrito para símbolos matemáticos
\usepackage{subfig} % Subfiguras dentro de figuras
\usepackage{float} % Controle preciso de posicionamento de figuras e tabelas
\usepackage{mathrsfs} % Fonte para matemática
\usepackage{lscape} % Páginas em modo paisagem
\usepackage{booktabs} % Linhas de qualidade superior para tabelas
\usepackage{csvsimple} % Importação de dados CSV
\usepackage{pdfpages} % Inclusão de páginas PDF
\usepackage{multirow} % Células de múltiplas linhas em tabelas
\usepackage{icomma} % Uso de vírgulas em expressões matemáticas
\usepackage{listings} % Inclusão de códigos fonte
\usepackage[font=itshape,vskip=1em,indentfirst=false]{quoting} % Citações com formatação específica (vskip controla o espaçamento vertical antes e depois)
\usepackage{makecell} % Personalização de células em tabelas
\usepackage{fancyhdr} % Configurações avançadas de cabeçalhos e rodapés
\usepackage{blindtext} % Texto de exemplo
\usepackage{titlecaps} % Capitalização de títulos nas referências
\usepackage{ulem}
\usepackage[style=brazilian]{csquotes}