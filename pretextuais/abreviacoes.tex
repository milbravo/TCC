% Siglas e abreviaturas utilizadas no texto devem ser apresentadas em uma lista 
% alfabética seguida de sua grafia por extenso. A primeira vez que a sigla 
% aparece no texto deve-se pontuar a expressão por extenso, seguida da sigla 
% entre parênteses; nas demais vezes, utiliza-se somente a sigla, inserida
% diretamente no texto.

\newpage
\chapter*{Lista de Siglas e Símbolos} % Inglês: List of Abbreviations and Symbols

	\section*{Siglas}
	
		\begin{itemize}[labelwidth=5em,leftmargin=\dimexpr\labelwidth+\labelsep\relax,align=left]
			\item[LC] Low-code
			\item[RPA] Robot Process Automation
			\item[RAG] Retrieval-Augmented Generation
			\item[ES] Engenharia de Sistemas
			\item[SoI] System of Interest
			\item[SoS] System of Systems
			\item[T\&A] Testes e Avaliações
			\item[IA] Inteligência Artificial
		\end{itemize}
	
		\thispagestyle{empty}

	% A princípio, essa parte não é necessária.
	% \section*{Símbolos}
	
	% 	\thispagestyle{empty}
	
	% 	\begin{itemize}[labelwidth=4em,leftmargin=\dimexpr\labelwidth+\labelsep\relax,align=left]
	% 		\item[$\epsilon$] Permissividade complexa [F/m + $j\Omega$/m]
	% 		\item[$\epsilon_r$] Permissividade relativa
	% 		\item[$\theta$] Ângulo da coordenada polar [rad]
	% 		\item[$\lambda_b $] Comprimento de onda de fundo [m]
	% 		\item[$\sigma$] Condutividade [$\Omega$/m]
	% 		\item[$\phi$] Ângulo de incidência [rad]
	% 		\item[$\mathbf{E}$] Vetor de intensidade elétrica [V/m]
	% 		\item[$E_z$] Componente $z$ do vetor de intensidade elétrica [V/m]
	% 		\item[$k$] Número de onda [1/m]
	% 		\item[$\mathbb{R}$] Conjunto dos números reais
	% 		\item[$\mathbf{r}$] Vetor posição no espaço 3D [m]
	% 		\item[$x, y, z$] Coordenadas cartesianas [m]
	% 		\item[$V$] Espaço tridimensional
	% 	\end{itemize}
	
		% \thispagestyle{empty}
	
	