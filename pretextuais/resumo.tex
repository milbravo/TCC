% O resumo é elemento obrigatório e consiste em um texto 
% conciso que representa os pontos relevantes do texto, devendo 
% conter de 150 a 500 palavras. Ele deve abarcar o objeto da 
% pesquisa, os objetivos, a metodologia, os resultados e a 
% conclusão.
% Abaixo do resumo, localizam-se as palavras-chaves que são 
% termos indicativos do conteúdo do trabalho e devem ser 
% precedidos da expressão Palavras-chave. São redigidas com a 
% inicial minúscula, separadas entre si com ponto-e-vírgula e finalizadas 
% com ponto final. 

\chapter*{Resumo}

	\noindent Este trabalho aplicou os conceitos e boas práticas da Engenharia de Sistemas (ES) para revisar e melhorar o ciclo de vida de um serviço de
	desenvolvimento de software \textit{low-code} em um ambiente organizacional real. A análise sob a ótica da ES permitiu uma tomada de decisão mais embasada e a proposição
	de modificações eficazes, especialmente nas fases de definição e validação de conceito e na rastreabilidade dos componentes do sistema.

	\noindent Foram implementadas mudanças no processo e introduzidas ferramentas de apoio, o que resultou em maior eficiência e detalhamento na distribuição do esforço por tarefa.
	A aplicação prática dessas modificações	em dois sistemas reais mostrou uma redução na concentração de esforço em tarefas isoladas, uma maior uniformidade na
	estimativa de esforço e melhor engajamento das áreas clientes durante a validação.

	\noindent A ferramenta desenvolvida para garantir a rastreabilidade dos sistemas contribuiu para aumentar a precisão das estimativas de esforço, identificando
	dependências e desafios técnicos que, de outra forma, poderiam ser subestimados. Isso proporcionou um planejamento mais realista, melhor gerenciamento da
	carga de trabalho e maior capacidade de gestão de riscos técnicos nas fases de manutenção e melhorias.

	\noindent Os resultados evidenciam a importância de um ciclo de vida bem estruturado e reforçam a aplicabilidade da Engenharia de Sistemas em contextos além
	dos tradicionais, alinhando-se às tendências atuais de automação e uso de inteligência artificial.

	\noindent Este estudo contribui para o avanço do conhecimento em Engenharia de Sistemas ao demonstrar, de forma prática, seu potencial para transformação
	organizacional e adaptação metodológica em ambientes de desenvolvimento de software.


	\vspace{5mm}
	
	% Lembre-se: cada palavra-chave começando em minúscula e separadas por 
	% ponto-e-vírgula.
	\noindent\textbf{Palavras-chave}: engenharia de sistemas; ciclo de vida; low-code;validação de conceito; definição de conceito; rastreabilidade; requisitos; desenvolvimento de software.
	
	\thispagestyle{empty}