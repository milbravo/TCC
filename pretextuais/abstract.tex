\chapter*{Abstract}

	\noindent This work applied the concepts and best practices of Systems Engineering (SE) to review and improve the lifecycle of a low-code software development service in a real organizational environment. The analysis from the SE perspective enabled more informed decision-making and the proposal of effective modifications, especially in the phases of concept definition and validation, as well as in the traceability of system components.

	\noindent Process changes were implemented and support tools introduced, resulting in greater efficiency and more detailed distribution of effort per task. The practical application of these modifications in two real systems demonstrated a reduction in effort concentration on isolated tasks, greater uniformity in effort estimation, and better engagement of client areas during validation.

	\noindent The tool developed to ensure system traceability contributed to increased accuracy in effort estimation by identifying dependencies and technical challenges that might otherwise have been underestimated. This enabled more realistic planning, improved workload management, and enhanced technical risk management during maintenance and improvement phases.

	\noindent The results highlight the importance of a well-structured lifecycle and reinforce the applicability of Systems Engineering beyond traditional contexts, aligning with current trends in automation and the use of artificial intelligence.

	\noindent This study contributes to advancing knowledge in Systems Engineering by demonstrating, in practical terms, its potential for organizational transformation and methodological adaptation in software development environments.

	
	\vspace{5mm}
	
	% Remember: each keyword starting in lowercase and separated by semicolon.
	\noindent\textbf{Keywords}: systems engineering; lifecycle; low-code; concept validation; concept definition; traceability; requirements; software development.

	
	\thispagestyle{empty}
	